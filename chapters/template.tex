% 模板文件template.tex
% 所有文件编译的时候都有可能默认编译了GCET-main.tex,这时候请点击VSCode右上角的绿色三角形,手动选择第二项Subfiles进行编译,之后每次保存编译的就是子文件了

\documentclass[../GCET-main.tex]{subfiles}
% 这里是subfiles包调用原始文件GCET-main.tex的语句,使得这个文件可以直接使用原始文件中所有宏包。
% ../代表上层文件夹,在章文件中应为../../GCET-main.tex,节文件中应为../../../GCET-main.tex。
% 注意必须调用到原始文件GCET-main.tex,否则报错。
% 注意这个调用是单向的(比如这个文件template.tex就没有写入主文件中),只是为了让你能单独编译一章或一节,防止文件过大编译速度太慢。最终放到主文件中还需要在主文件中使用\subfile{chapters/chapter1/1气体、液体和固体.tex}才能插入文件
% subfiles调用无法保留章和节的计数器,所以单独编译时章的编号会是1,节的编号会是0.1,但整本书中编号不会出问题

\begin{document} % 仍然需要本语句

% chapter,section,subsection记得用\label{}做好引用,chapter用\label{1}section用\label{1.1}subsection用\label{1.1.1}
% LaTeX自带的环境,比如经常需要用到的figure和table,需要在\caption之后加\label{figure1.1}或\label{table1.1}
% 几个特殊环境(下面会介绍的色块)会自动编号,只需要知道编号方式是环境名称跟序号就可以,如definition1.1
%所有引用统一用\autoref{1},测试了好多种方法还是这个最好用
% 记住在编第几章第几节,忘记了就用更大的文件编译,注意引用需要在同一个文件包括被引用的一方时才能被引用到
% 引用需要编译两次才能得到

\chapter{章节标题}\label{1} % 这个要在chapter级别的文件写
章节下面可以写东西
\section{小节标题}\label{1.1} % 在section级别的文件写下面的东西
小节下面可以写东西
\subsection{小小节标题}\label{1.1.1}
小小节下面还是可以写东西

\begin{figure}[htbp] % 图片环境
    \centering % 居中,不想居中可以不要
    \includegraphics[width=0.5\textwidth]{金盏花488.jpg} % 我在images文件夹里面放了一张金盏花488.jpg,图片还没做好的时候可以用这个先顶替一下
    \caption{金盏花488} % 标题
    \label{figure1.1} % \label一定要放在\caption之后
\end{figure}

\begin{table}[htbp] % 表格环境
    \centering % 居中,不想居中可以不要
    \begin{tabular}{|c|c|} % 正常写表格
        \hline
        序号 & 内容 \\
        \hline
        1 & 测试 \\
        \hline
    \end{tabular}
    \caption{第一张示例表} % 标题
    \label{table1.1} % \label一定要放在\caption之后
\end{table}

% GCET-book.cls中用tcolorbox宏包新定义了色块,再在这个基础上使用不同颜色定义了新的环境,可以跨多页书写
% 新定义的环境目前有以下几种:
\begin{definition} % 定义 蓝色 自动编号
    理想气体:分子间没有相互作用力、分子本身没有体积的气体
\end{definition}

\begin{concept} % 概念 蓝色 自动编号
化学键的概念:
\end{concept}

\begin{formula} % 公式 紫色 自动编号
\[pV_m=RT\]
\[pM=\rho RT\]
\end{formula}

\begin{reaction} % 反应方程式 黄色 自动编号
\[\ce{2H_2 +O2 ->[\text{点燃}]2H2O}\]
\end{reaction}

\begin{exercise} % 例题 红色 自动编号
求方程$\hat{H}\psi=E\psi$的多电子解
\end{exercise}

\begin{forexample} % 例子 青色 不编号
常见的中心原子$\ce{sp^3}$杂化的分子:\[\ce{CH4},\ \ce{NH3},\ \ce{H2O}\]
\end{forexample}

\begin{proof} % 证明(可能不太用到) 绿色 不编号
    ?这对吗
\end{proof}

\begin{derivation} % 推导 绿色 不编号
由理想气体状态方程:
\[pV=nRT\]
移项得:
\[p\dfrac{V}{n}=RT\]
即:\[pV_\mathrm{m}=RT\]
所以\[aaa\]
这个可以跨页吗
\end{derivation}

\begin{answer} % 解答 绿色 不编号
目前薛定谔方程已经存在了一百年,但是仍然没有人解出多电子情况下的解析解,数学工具的落后限制了化学的发展。
\end{answer}

编译两次测试一下引用效果\autoref{1},\autoref{figure1.1},\autoref{table1.1},
\autoref{concept1.1}

% 此外其他一些LaTeX自带的原始内容也有修改,比如enumerate环境多层嵌套可以自动改变编号样式,第一层是1.第二层是(1)第三层是i.:
\begin{enumerate}
    \item 第一层是1.
    \begin{enumerate}
        \item 第二层是(1)
        \begin{enumerate}
            \item 第三层是i.
            \item 这是ii.
            \item 这是iii.
            \item 这是iv.
        \end{enumerate}
        \item 这是(2)
    \end{enumerate}
    \item 这是2.
\end{enumerate}

% 所以一般情况下用到enumerate请不要调整参数

% 选择题可以用tasks宏包中新环境写,比如:
\begin{exercise}
下列分子中存在$\pi$键的是:
\begin{tasks}[label=\Alph*.](4) % label参数和enumerate环境类似,后面()中参数为每一行的选项数量
    \task $\ce{C2H6}$
    \task $\ce{CH4}$
    \task $\ce{H2SO4}$
    \task $\ce{PCl3}$
\end{tasks}
\end{exercise}

\end{document}