\documentclass[../../../GCET-main.tex]{subfiles}

\begin{document}

\phantomsection
\addcontentsline{toc}{section}{习题}
\section*{习题}
\subsection*{判断题}
\begin{enumerate}
    \item 液体和气体都可以压缩,固体不能压缩。\hfill (\qquad) % False
    \item 一个气体分子满足理想气体状态方程。\hfill (\qquad) % False
    \item 符合范德华方程的气体称为范德华气体。\hfill (\qquad) % True
    \item 目前已经找到了一个真实气体状态方程可以描述所有真实气体。\hfill (\qquad) % False
    \item 高于临界温度的状态就是临界状态。\hfill (\qquad) % False
    \item 水可以同时沸腾和结冰。\hfill (\qquad) % True
    \item 气体温度在临界温度以上,只要压强足够大,仍然可以液化。\hfill (\qquad) % False
    \item 对某温度下的水蒸气加压,水蒸气可以先变成冰再变成水。\hfill (\qquad) % True
    \item 气体转化为液体可以不经过相变。\hfill (\qquad) % True
    \item 增大压强,液体凝固点一定升高。\hfill (\qquad) % False
    \item 装满水的水箱内部的水没有饱和蒸气压。\hfill (\qquad) % False
    \item 液体的饱和蒸气压接近外压时汽化非常剧烈,被称为沸腾。\hfill (\qquad) % True
    \item 任何浓度的溶液都满足拉乌尔定律。\hfill (\qquad) % False
    \item 质量摩尔浓度相等的的蔗糖溶液和$\ce{NaCl}$溶液熔沸点几乎一致。\hfill (\qquad) % False
    \item 表面张力$\sigma$在国际单位制下的单位是$\mathrm{kg\cdot m/s^2}$。\hfill (\qquad) % False
    \item 小液滴内部的压强大于外界气体的压强。\hfill (\qquad) % True
    \item 水浸润玻璃,所以水在细玻璃管内会上升一段高度。\hfill (\qquad) % True
    \item 为了保存地下的水分,我们需要用磙子压紧土壤,防止水分蒸发。\hfill (\qquad) % False
    \item 实验室中加热液体到沸点以上时发现忘记加沸石,应该马上补加。\hfill (\qquad) % False
    \item 晶体具有各向同性。\hfill (\qquad) % False
    \item 冰的结构和金刚石的结构类似,它们属于同一种晶体。\hfill (\qquad) % False
    \item 金属在常温下都是固体。\hfill (\qquad) % False
    \item 离子晶体的熔点都很高。\hfill (\qquad) % False
\end{enumerate}
\subsection*{简答题}
\begin{enumerate}[start=24]
    \item 某容器中含$112\ \mathrm{g}\ \ce{N2}$和$32\ \mathrm{g}\ \ce{O2}$,$300\ \mathrm{K}$时容器体积为$0.5\ \mathrm{m^3}$,将容器内所有气体视为理想气体,气体常数$R$取$8.314\ \mathrm{J/mol\cdot K}$,计算:
    \begin{enumerate}
        \item 容器内气体的总压$p$和$\ce{O2}$的分压$p_{\ce{O2}}$;
        \item 容器内气体的平均摩尔质量$\overline{M}$(用总质量除以总物质的量)。
        \item 容器内气体的密度$\rho$。
    \end{enumerate}
    \item 下面展示的是$\ce{He}$的相图。\\
    \begin{center}
        \includegraphics[width=0.4\textwidth]{1e25.png}
    \end{center}
    请回答下列问题(数值精确到1即可):
    \begin{enumerate}
        \item 超流体$\ce{He}$-I能存在的最低温度是多少?
        \item 固体能存在的最低压强是多少?
        \item 气体$\ce{He}$能否凝华?
    \end{enumerate}
    \item 一个标准大气压下$101.325\ \mathrm{kPa}$水的沸点为$100\ \mathrm{℃}=373.15\ \mathrm{K}$,$20\ \mathrm{℃}$时水的饱和蒸气压为$2.339\ \mathrm{kPa}$,假设水的摩尔蒸发焓$\Delta H_\mathrm{vap}$不随温度变化,回答下列问题:
    \begin{enumerate}
        \item 求水的摩尔蒸发焓$\Delta H_\mathrm{vap}$;
        \item 在某地烧水到$90\ \mathrm{℃}$就会沸腾,求当地的大气压大小(单位$\mathrm{kPa}$)。
    \end{enumerate}
    \item 苯和甲苯组成的溶液可以看作理想溶液。$20\ \mathrm{℃}$下将$3\ \mathrm{mol}$苯和$2\ \mathrm{mol}$甲苯混合,与溶液平衡的气体中苯和甲苯的分压之比为$p_\text{苯}:p_\text{甲苯}=5:1$,求苯和甲苯的饱和蒸气压之比$p_\text{苯}^*:p_\text{甲苯}^*$。
    \item 纯水的凝固点为$0\ \mathrm{℃}$,水的凝固点下降常数$K_\mathrm{f}=1.86\ \mathrm{K\cdot kg/mol}$,现测得含$0.02\ \mathrm{mol}$未知离子化合物的$2\ \mathrm{kg}$水溶液凝固点下降了$0.0744\ \mathrm{℃}$,请回答下列问题:
    \begin{enumerate}
        \item $1\ \mathrm{mol}$此离子化合物能在水中电离出多少离子?
        \item 在$300\ \mathrm{K}$下将此离子化合物配制成$0.01\ \mathrm{mol/L}$的水溶液,此溶液的渗透压$\Pi$是多少?
    \end{enumerate}
    \item 已知地球半径为$6400\ \mathrm{km}$,$20\ \mathrm{℃}$下水的表面张力为$72.75\ \mathrm{mN/m}$。假设地球上的海洋表层平均温度为$20\ \mathrm{℃}$,如果把地球上的海洋表层看作弯曲液面,此弯曲液面的附加压强是多少?相比于大气压$p^0=101.325\ \mathrm{kPa}$,这部分压强对地球影响如何?
\end{enumerate}

\end{document}