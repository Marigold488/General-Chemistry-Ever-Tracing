\documentclass[../../GCET-main.tex]{subfiles}

\begin{document}

\chapter{气体、液体和固体}\label{1}

化学研究的物质通常以气态、液态和固态的形式存在。我们根据是否具有固定的形状和体积来区分物质的状态。
\begin{definition}
    固态:有固定的形状和固定的体积\par
    液态:无固定的形状但是有固定的体积\par
    气态:既没有固定的形状也没有固定的体积
\end{definition}\par
相较于液体和固体,气体体系较容易研究,用来描述气体状态的物理量(温度$T$、体积$V$和压强$p$)容易使用仪器测量和改变。我们先从气体开始研究,随后进一步研究各种物质之间互相转化的方式——相变,主要研究与气体相关的相变。
\subfile{section1/1.1气体的状态方程.tex}
\subfile{section2/1.2相变.tex}

\end{document}