\documentclass[../../GCET-main.tex]{subfiles}

\begin{document}

\chapter{气体、液体和固体}\label{1}

化学研究的物质通常以气态、液态和固态的形式存在。我们根据是否具有固定的形状和体积来区分物质的状态。
\begin{definition}
    固态:有固定的形状和固定的体积\par
    液态:无固定的形状但是有固定的体积\par
    气态:既没有固定的形状也没有固定的体积
\end{definition}\par
相较于液体和固体,气体体系较容易研究,用来描述气体状态的物理量(温度$T$、体积$V$和压强$p$)容易使用仪器测量和改变。我们先从气体开始研究,随后进一步研究各种物质之间互相转化的方式——相变,主要研究与气体相关的相变。
\subfile{section1/1.1气体的状态方程.tex}
\subfile{section2/1.2相变.tex}
\subfile{section3/1.3饱和蒸气压.tex}
\subfile{section4/1.4溶液.tex}
\subfile{section5/1.5液体的表面张力.tex}
\subfile{section6/1.6固体.tex}
\phantomsection
\addcontentsline{toc}{section}{本章小结}
\section*{本章小结}
本章我们通过相图这一研究相变的重要工具串联起整章内容。我们在\autoref{1.2}中学习使用相图判断物质在特定温度、压强下的状态以及相变发生的条件,在\autoref{1.3}中通过\autoref{formula1.8}克劳修斯-克拉贝龙方程和\autoref{formula1.9}克拉贝龙方程绘制了相图的三条两相平衡线,又在\autoref{1.4}中利用稀溶液相图中两相平衡线位置的变化推导得出稀溶液的依数性。\par
除此之外,在本章的开头,我们介绍了气体,为后续小节的推导打下了基础;在本章的结尾,我们介绍了液体表面的性质,解释了生活与生产中的一些特殊现象;本章提及固体的部分较少,因为在\textcolor{blue}{(章节)}中我们可能会介绍更多与晶体相关的知识。\par
本章的推导过程中有较多的近似处理,比如在稀溶液中认为溶液可以看作都由液体组成,两相平衡线位置变化时认为各种温度变化的幅度不大。这些近似处理实际上用到了数学、物理中的等价无穷小替换、小量近似等方法,对微积分知识的要求较高。此外,我们还在研究液体的表面张力时用到了力的微分与环积分的知识,这是数学、物理知识在化学中的应用。熟悉这些推导过程对理解数学、物理中其他利用微积分知识的推导和计算非常有帮助。
\subfile{exercise/1exercise.tex}
\subfile{exercise/1answer.tex}

\end{document}