\documentclass[../../../GCET-main.tex]{subfiles}

\begin{document}

\section{固体}\label{1.6}
固体可以分为\textbf{晶体}和\textbf{非晶体}两类。
\begin{definition}
    \textbf{晶体}具有天然的、规则的几何形状,具有确定的熔点,具有\textbf{各向异性}。\par
    \textbf{非晶体}没有规则的外形,没有确定的熔化温度,具有\textbf{各向同性}。\par
    \textbf{各向异性}:导热性、导电性等物理性质随方向改变\par
    \textbf{各向同性}:导热性、导电性等物理性质不随方向改变\par
    判断是晶体还是非晶体的方法:能否产生X-射线衍射现象
\end{definition}\par
石英、云母、明矾、食盐、硫酸铜、味精、雪花等是晶体,玻璃、蜂蜡、松香、沥青、橡胶等是非晶体。晶体和非晶体可以在一定条件下互相转化。\par
但存在这么一类物质,它们没有规则的形状,但具有确定的熔点,比如金属和蔗糖受潮之后粘在一起形成的糖块,这些物质也是晶体,属于晶体中的\textbf{多晶体},多晶体由\textbf{单晶体}组成。
\begin{definition}
    \textbf{单晶体}:只由一个晶粒构成的晶体\par
    \textbf{多晶体}:许多大小、取向各不相同的单晶体随机或半随机堆积而成的晶体
\end{definition}\par
得到晶体一般有三种途径:
\begin{enumerate}
    \item 熔融态物质凝固($\ce{S}$)
    \item 气态物质凝华($\ce{I2}$)
    \item 溶质从溶液中析出($\ce{CuSO4.5H2O}$)
\end{enumerate}

\subsection{晶体基础}\label{1.6.1}
晶体可以分为\textbf{分子晶体}、\textbf{共价晶体}、\textbf{金属晶体}、\textbf{离子晶体}、\textbf{过渡晶体}、\textbf{混合型晶体}。
\begin{definition}
    \textbf{分子晶体}:只含分子的晶体\par
    \textbf{共价晶体}:有共价键三维骨架结构的晶体\par
    \textbf{金属晶体}:金属阳离子通过金属键相互结合形成的晶体\par
    \textbf{离子晶体}:阳离子和阴离子相互作用形成的晶体\par
    \textbf{过渡晶体}:性质介于两种晶体之间的晶体\par
    \textbf{混合型晶体}:晶体内部存在两种或两种以上不同类型的作用力的晶体
\end{definition}\par
不同类型的晶体内部作用力类型不同,导致不同类型的晶体性质不同,下面一一介绍。
\subsubsection{分子晶体}
分子晶体中分子通过分子间作用力相互吸引,熔点较低,硬度很小。如果某种分子晶体中分子间作用力包含氢键,那么这种分子晶体熔点会高于结构相似的分子晶体。\par
冰的晶体中含有氢键,由于氢键具有方向性与饱和性,冰的晶体中每个水分子周围只有四个水分子,如\autoref{figure1.9}所示。这一排列使冰晶体中水分子的空间利用率不高,导致冰的密度比液态水小。
\begin{forexample}
    常见的分子晶体:
    \begin{enumerate}
        \item 大部分非金属氢化物:如$\ce{H2O}$、$\ce{H2S}$、$\ce{NH3}$、$\ce{HCl}$、$\ce{CH4}$等;
        \item 部分非金属单质:如卤素$\ce{X2}$、$\ce{O2}$、$\ce{S8}$、$\ce{N2}$、白磷$\ce{P4}$、$\ce{C_{60}}$等;
        \item 部分非金属氧化物:如$\ce{CO2}$、$\ce{P4O6}$、$\ce{P4O_{10}}$、$\ce{SO2}$等;
        \item 小部分盐:如$\ce{AlCl3}$、$\ce{FeCl3}$、$\ce{BeCl2}$、$\ce{HgCl2}$等;
        \item 稀有气体;
        \item 几乎所有的酸;
        \item 绝大多数有机物。
    \end{enumerate}
\end{forexample}
\begin{figure}[h]
    \centering
    \includegraphics[width=0.3\textwidth]{1.6.1冰的结构.png}
    \includegraphics[width=0.3\textwidth]{1.6.1金刚石的三维骨架结构.png}  
    \caption{冰的结构和金刚石的三维骨架结构}
    \label{figure1.9}
\end{figure}
\subsubsection{共价晶体}
共价晶体中原子通过共价键直接结合,硬度高,熔点高。比如金刚石中碳原子通过共价键形成三维骨架结构,如\autoref{figure1.9}所示。
\begin{forexample}
    常见的共价晶体:
    \begin{enumerate}
        \item 某些单质:如金刚石$\ce{C}$、$\ce{B}$、$\ce{Si}$、$\ce{Ge}$、灰锡$\ce{Sn}$等;
        \item 某些非金属化合物:如$\ce{SiO2}$、金刚砂$\ce{SiC}$、$\ce{Si3N4}$等;
        \item 某些金属氧化物:$\ce{Al2O3}$等。
    \end{enumerate}
\end{forexample}
\subsubsection{金属晶体}
金属(除汞外)在常温下都是晶体。金属晶体中金属原子之间通过金属键相互结合,金属键的强度差别很大。有的金属比如钠熔点很低、硬度较小,用小刀就可以切开。钨熔点超过$3000\ \mathrm{℃}$,铬是硬度最大的金属。金属具有良好的延展性、导电性、导热性。\par
金属形成合金后仍然是金属晶体,大多数合金以一种金属为主。合金的性质与纯金属不同,大部分情况下合金的熔点低于形成合金的几种金属。\par
\textbf{电子气理论}可以解释金属键、延展性、导电性及导电性随温度的变化。该理论认为金属原子脱落下来的价电子被所有原子共用,形成遍布整块晶体的“电子气”。“电子气”可以在金属受外力变形时起“润滑剂”的作用,使金属晶体有良好的延展性。“电子气”可以在电场下定向移动,使金属晶体有良好的导电性,自由电子在热的作用下会与金属原子频繁碰撞,导致金属的电导率随温度升高而降低。\par
\textbf{能带理论}也可以解释金属键,同时可以解释导体、半导体、绝缘体的导电性。能带理论基于分子轨道理论,这里不作介绍。
\begin{forexample}
    常见的金属晶体:
    \begin{enumerate}
        \item 所有金属单质;
        \item 绝大多数合金:如以铁为主要成分的碳钢、锰钢、不锈钢,以铜为主要成分的黄铜、青铜、白铜等。
    \end{enumerate}
\end{forexample}
\subsubsection{离子晶体}
离子晶体中阳离子和阴离子通过离子键相互作用结合,离子晶体的性质与离子键强度有关。离子半径越小、电荷数越多,离子键越强。$\ce{NaCl}$晶体中离子键强度较大,熔沸点较高,硬度较大,难以压缩;含有有机基团的离子晶体中离子键强度较小,熔沸点较低,硬度较小,在室温下甚至可以呈液态,称为\textbf{离子液体}。\par
离子晶体中不只有阳离子和阴离子,还可能存在电中性分子。离子晶体中不只存在离子键,还可能存在共价键、分子间作用力,但贯穿整个离子晶体的主要作用力仍然是离子键。
\begin{forexample}
    常见的离子晶体:
    \begin{enumerate}
        \item 一部分金属氧化物:如$\ce{Na2O}$、$\ce{MgO}$等;
        \item 大部分盐:如$\ce{NaCl}$、$\ce{MgCl2}$、$\ce{CuSO4}$、$\ce{NH4Cl}$等;
        \item 部分碱:如$\ce{NaOH}$、$\ce{Mg(OH)_2}$、$\ce{Ba(OH)_2}$等。
    \end{enumerate}
\end{forexample}
\subsubsection{过渡晶体}
过渡晶体是性质介于两种晶体之间的晶体,性质更接近两种晶体中的哪一种,就当作是哪一种晶体。常见的过渡晶体有离子晶体和共价晶体之间的过渡、离子晶体和分子晶体之间的过渡。\par
\subsubsection{混合型晶体}
混合型晶体中存在两种或两种以上不同类型作用力,比如石墨中同时存在共价键和分子间作用力,形成层状结构,如\autoref{figure1.10}所示。
\begin{figure}[h]
    \centering
    \includegraphics[width=0.3\textwidth]{1.6.1石墨的层状结构.png} 
    \caption{石墨的层状结构}
    \label{figure1.10}
\end{figure}

\end{document}