\documentclass[../../../GCET-main.tex]{subfiles}

\begin{document}

\phantomsection\label{2exercise}
\addcontentsline{toc}{section}{习题}
\section*{习题}
\subsection*{判断题}
\begin{enumerate}
    \item 经过一个循环后,$Q$、$W$、$\Delta U$、$\Delta H$、$\Delta S$、$\Delta G$一定都等于0。\hfill (\qquad) % False
    \item 现实世界中存在可逆过程。\hfill (\qquad) % False
    \item 理想气体的恒温过程中,体系与环境有可能只有功交换,没有热交换。\hfill (\qquad) % False
    \item 一定量的理想气体可以吸收热量而不改变内能。\hfill (\qquad) % True
    \item 与外界无物质交换的绝热恒容容器中发生任何过程,都有$\Delta U=0$。\hfill (\qquad) % False
    \item 在发生相变化和化学反应的过程中$\Delta U$、$\Delta H$仍然可以直接用对应的$\Delta U=C_\mathrm{V}\Delta T$和$\Delta H=C_\mathrm{p}\Delta T$计算。\hfill (\qquad) % False
    \item 气体分子数不变时,恒压反应热和恒容反应热的大小相同。\hfill (\qquad) % True
    \item 在孤立体系中,$\Delta S=0$不一定能说明该过程为可逆过程。\hfill (\qquad) % False
    \item 焓和熵都没有具体数值,我们不能知道某物质的焓和熵是多少,只能计算过程的焓变和熵变。\hfill (\qquad) % False
    \item 封闭体系在恒温恒压且非体积功为零的情况下,若$\Delta G=0$,可以认为该过程为可逆过程。\hfill (\qquad) % True
    \item 石墨的标准摩尔生成焓、标准摩尔熵、标准摩尔生成吉布斯自由能都是0。\hfill (\qquad) % False
    \item 平衡状态的反应商$J^{eq}$等于标准平衡常数$K^0$。\hfill (\qquad) % True
    \item 一定条件下某反应的$K^0<1$,说明该条件下此反应不能自发进行。\hfill (\qquad) % False
    \item $\mathrm{pH}+\mathrm{pOH}=14$在任意温度下均成立。\hfill (\qquad) % False
    \item $\mathrm{p}K_\mathrm{a}$越大,酸性越强。\hfill (\qquad) % False
    \item 在$\ce{HAc}$溶液中加入纯$\ce{NaAc}$会抑制$\ce{HAc}$的电离。\hfill (\qquad) % True
\end{enumerate}
\subsection*{简答题}
\begin{enumerate}[start=17]
    \item $\ce{SiH2}$是$\ce{SiH4}$和$\ce{Si2H6}$热分解的关键中间体,现已知
    \[\Delta_\mathrm{f}H_\mathrm{m}^0(\ce{SiH2})=+274\ \mathrm{kJ/mol}\]
    \[\Delta_\mathrm{f}H_\mathrm{m}^0(\ce{SiH4})=+34.3\ \mathrm{kJ/mol}\]
    \[\Delta_\mathrm{f}H_\mathrm{m}^0(\ce{Si2H6})=+80.3\ \mathrm{kJ/mol}\]
    计算下列反应的标准化学反应焓:
    \begin{enumerate}
        \item $\ce{SiH4(g) -> SiH2(g) + H2(g)}$
        \item $\ce{Si2H6(g) -> SiH2(g) + SiH4(g)}$
    \end{enumerate}
    \item 已知玻尔兹曼常数$k=1.38\times10^{-23}\ \mathrm{J/K}$,如果一个过程$\Delta S=0.001\ \mathrm{J/K}$,那么从始态到终态,体系的微观状态数扩大几倍?
    \item 定性判断下列反应的$\Delta_\mathrm{r}S_\mathrm{m}^0$的符号
    \begin{enumerate}
        \item $\ce{Zn(s) + 2HCl(aq) -> ZnCl2(aq) + H2(g)}$
        \item $\ce{CaCO3(s) -> CaO(s) + CO2(g)}$
        \item $\ce{NH3(g) + HCl(g) -> NH4Cl(s)}$
        \item $\ce{CuO(s) + H2(g) -> Cu(s) + H2O(l)}$
    \end{enumerate}
    \item 已知分解反应:
    \[\ce{SnS2(s) -> S(g) + SnS(s)}\]
    \begin{center}
        \begin{tabular}{cccc}
            \toprule
            分解温度附近数值 & $\ce{SnS2(s)}$ & $\ce{S(g)}$ & $\ce{SnS(s)}$ \\
            \midrule
            $\Delta_\mathrm{f}H_\mathrm{m}^0(\mathrm{kJ/mol})$ & -116.14 & 76.84 & -75.69 \\
            $S_\mathrm{m}^0(\mathrm{J/(mol\cdot K)})$ & 160.32 & 135.36 & 136.10 \\
            \bottomrule
        \end{tabular}
    \end{center}
    假设$\Delta_\mathrm{f}H_\mathrm{m}^0$和$S_\mathrm{m}^0$不随温度变化,反应在标准状态下进行,通过计算确定该分解反应自发进行的温度范围。
    \item 下列反应在$298.15\ \mathrm{K}$、$100\ \mathrm{kPa}$下进行:
    \[\ce{$\dfrac{1}{2}$Br2(g) + $\dfrac{1}{2}$H2(g) <=> HBr(g)}\]
    已知$\Delta_\mathrm{f}G_\mathrm{m}^0(\ce{Br2(g)})=3.08\ \mathrm{kJ/mol}$,$\Delta_\mathrm{f}G_\mathrm{m}^0(\ce{HBr(g)})=-53.45\ \mathrm{kJ/mol}$,此时反应容器中$p(\ce{Br2})=40\ \mathrm{kPa}$,$p(\ce{H2})=10\ \mathrm{kPa}$,回答下列问题:
    \begin{enumerate}
        \item 求此反应的$\Delta_\mathrm{r}G_\mathrm{m}^0$;
        \item 求此反应的$K^0$;
        \item 判断此条件下反应能否自发进行。
    \end{enumerate}
    \item $10\ \mathrm{dm^3}$氩气由$273\ \mathrm{K}$和$506.625\ \mathrm{kPa}$,经绝热可逆膨胀到$101.325\ \mathrm{kPa}$,计算终态温度$T$和此过程的$Q$、$\Delta U$。
    \item 已知乙醇($M=46\ \mathrm{g/mol}$)的$C_\mathrm{p,m}=111.46\ \mathrm{J/(mol\cdot K)}$且可认为不随温度变化,150克乙醇($22\ \mathrm{℃}$)与200克乙醇($56\ \mathrm{℃}$)在绝热容器中恒压下混合,求体系熵变。
    \item 已知苯在$101.325\ \mathrm{kPa}$的沸点是$353.25\ \mathrm{K}$,摩尔蒸发焓$\Delta_\mathrm{vap}H_\mathrm{m}=30.7\ \mathrm{kJ/mol}$。$1\ \mathrm{mol}$苯在$101.325\ \mathrm{kPa}$、$353.25\ \mathrm{K}$下蒸发,得到$101.325\ \mathrm{kPa}$、$353.25\ \mathrm{K}$的苯蒸气。液态苯的体积相对于气态苯可忽略不计,求该过程中的$Q$、$W$、$\Delta U$、$\Delta S$、$\Delta G$。
\end{enumerate}

\end{document}