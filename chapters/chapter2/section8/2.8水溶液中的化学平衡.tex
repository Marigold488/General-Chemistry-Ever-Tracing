\documentclass[../../../GCET-main.tex]{subfiles}

\begin{document}

\section{水溶液中的化学平衡**}\label{2.8}
电离平衡和沉淀溶解平衡是水溶液中的化学平衡,主要考虑水溶液中各离子的浓度对反应平衡的影响。由于标准浓度$c^0=1\ \mathrm{mol/L}$,我们将以$\mathrm{mol/L}$为单位的浓度标准化时相当于只是去掉了单位,因此计算起来比较方便。\par
电离平衡的对象是\textbf{弱电解质},即那些在水中无法完全电离的\textbf{电解质}。我们根据化合物在水中的电离能力把化合物分为以下几种。
\begin{definition}
    \textbf{电解质}:在水溶液中或熔融状态下能导电的化合物\par
    \textbf{非电解质}:在水溶液中和熔融状态下都不能导电的化合物\par
    \textbf{强电解质}:在水溶液中完全电离的电解质\par
    \textbf{弱电解质}:在水溶液中部分电离的电解质
\end{definition}\par
电解质和非电解质是化合物内部的分类,单质既不是电解质也不是非电解质。\par
强电解质的特点是在水溶液中能完全电离,$1\ \mathrm{mol/L}$的$\ce{NaCl}$能完全电离出$1\ \mathrm{mol/L}$的$\ce{Na+}$和$1\ \mathrm{mol/L}$的$\ce{Cl-}$。强电解质的电离可以视为不可逆反应,因此我们分析强电解质的电离时不需要用到电离平衡的知识。\par
弱电解质主要是弱酸和弱碱,水也是弱电解质,大部分盐都是强电解质。常见的弱酸有$\ce{CH3COOH}$、$\ce{H2CO3}$、$\ce{H2S}$等,常见的弱碱有$\ce{NH3.H2O}$。\par
水溶液的电离平衡常数一般都是标准平衡常数,但为了方便,后面的$K$不再加上标$0$,读者认为是标准平衡常数即可。
\subsection{水的电离平衡}\label{2.8.1}
水有微弱的导电性,水会电离出氢离子和氢氧根离子。
\begin{reaction}
    水的电离方程式:
    \[\ce{2H2O <=> H3O+ + OH-}\]
    用$\ce{H+}$代替$\ce{H3O+}$,上式可以简化为:
    \[\ce{H2O <=> H+ + OH-}\]
\end{reaction}\par
水的电离平衡常数又叫作水的\textbf{离子积常数},符号为$K_\mathrm{w}$。我们可以根据水的电离方程式写出$K_\mathrm{w}$的表达式:
\[K_\mathrm{w}=\dfrac{c(\ce{H+})}{c^0}\times\dfrac{c(\ce{OH-})}{c^0}\]
水作为溶剂,视为纯液体,不在\textbf{离子积常数}的表达式中。为了书写方便,我们用$[\ce{H+}]$代替$c(\ce{H+})$,用$[\ce{OH-}]$代替$c(\ce{OH-})$,这样$K_\mathrm{w}$的表达式可以写作:
\[K_\mathrm{w}=[\ce{H+}][\ce{OH-}]\]
电离反应一般是吸热反应,因此温度越高,$K_\mathrm{w}$越大,$K_\mathrm{w}$在各种温度下的数值如\autoref{figure2.5}所示。
\begin{table}[h]
    \centering
    \caption{$K_\mathrm{w}$在各种温度下的数值}
    \begin{tabular}{cccccc}
        \toprule
        $t/\mathrm{℃}$ & 0 & 10 & 25 & 50 & 100 \\
        \midrule
        $K_\mathrm{w}$ & $1.34\times10^{-15}$ & $2.92\times10^{-15}$ & $1.0\times10^{-14}$ & $5.47\times10^{-14}$ & $5.5\times10^{-13}$ \\
        \bottomrule
    \end{tabular}
    \label{table2.5}
\end{table}\par
我们可以用平衡常数的知识计算水中$\ce{H+}$和$\ce{OH-}$的浓度。
\begin{exercise}
    已知$25\ \mathrm{℃}$时$K_\mathrm{w}=10^{-14}$,计算$25\ \mathrm{℃}$下纯水中$\ce{H+}$的浓度。
\end{exercise}
\begin{answer}
    水的电离平衡方程为:
    \[K_\mathrm{w}=[\ce{H+}][\ce{OH-}]=10^{-14}\]
    水的电离方程式中,$\ce{H2O}$1:1电离生成$\ce{H+}$和$\ce{OH-}$,因此在纯水中有:
    \[[\ce{H+}]=[\ce{OH-}]\]
    所以:
    \[K_\mathrm{w}=[\ce{H+}]^2=10^{-14}\]
    解得:
    \[[\ce{H+}]=10^{-7}\]
    即水中$c(\ce{H+})=10^{-7}\ \mathrm{mol/L}$。
\end{answer}\par
我们一般用pH来表示溶液中$\ce{H+}$的浓度,对于$\ce{OH-}$也有pOH,对于平衡常数$K$也有对应的$\mathrm{p}K$的计算如下。
\begin{formula}
    \[\mathrm{pH}=-\lg[\ce{H+}]\]
    \[\mathrm{pOH}=-\lg[\ce{OH-}]\]
    \[\mathrm{p}K=-\lg K\]
\end{formula}\par
根据\autoref{exercise2.5},$25\ \mathrm{℃}$下纯水中$c(\ce{H+})=10^{-7}\ \mathrm{mol/L}$,所以$25\ \mathrm{℃}$下纯水的pH值为$-\lg 10^{-7}=7$,pOH的值也是$-\lg 10^{-7}=7$。$\mathrm{pH}<7$时,溶液呈酸性;$\mathrm{pH}>7$时,溶液呈碱性。pH与pOH有以下关系。
\begin{formula}
    pH与pOH的关系:
    \[\mathrm{pH}+\mathrm{pOH}=\mathrm{p}K_\mathrm{w}\]
    在$25\ \mathrm{℃}$下:
    \[\mathrm{pH}+\mathrm{pOH}=14\]
\end{formula}\par
这个公式非常好证明,只需要对$K_\mathrm{w}$的表达式两边取对数即可。
\begin{derivation}
    \qquad $K_\mathrm{w}$的表达式为:
    \[K_\mathrm{w}=[\ce{H+}][\ce{OH-}]\]
    两边取以$10$为底的对数得:
    \[\lg K_\mathrm{w}=\lg [\ce{H+}][\ce{OH-}]=\lg [\ce{H+}]+\lg [\ce{OH-}]\]
    两边取相反数得:
    \[-\lg K_\mathrm{w}=-\lg [\ce{H+}]-\lg [\ce{OH-}]\]
    即:
    \[\mathrm{p}K_\mathrm{w}=\mathrm{pH}+\mathrm{pOH}\]
\end{derivation}\par
在离子浓度的计算中经常需要用到\autoref{formula2.37}转换$[\ce{H+}]$和$[\ce{OH-}]$。
\subsection{弱酸与弱碱的电离平衡}\label{2.8.2}
一元弱酸$\ce{HA}$和一元弱碱$\ce{BOH}$只发生一步电离,它们的电离方程式和电离平衡常数表达式如下。
\begin{formula}
    一元弱酸的电离方程式:
    \[\ce{HA <=> H+ + A-}\]
    电离平衡常数:
    \[K_\mathrm{a}=\dfrac{[\ce{H+}][\ce{A-}]}{[\ce{HA}]}\]
    \[\mathrm{p}K_\mathrm{a}=-\lg K_\mathrm{a}\]
    一元弱酸的电离方程式:
    \[\ce{BOH <=> B+ + OH-}\]
    电离平衡常数:
    \[K_\mathrm{b}=\dfrac{[\ce{B+}][\ce{OH-}]}{[\ce{BOH}]}\]
    \[\mathrm{p}K_\mathrm{b}=-\lg K_\mathrm{b}\]
\end{formula}\par
电离平衡常数$K_\mathrm{a}$或$K_\mathrm{b}$越大,物质的酸性或碱性越强。常见弱酸和弱碱的电离平衡常数的数量级在$10^{-2}$到$10^{-14}$之间,我们可以依据电离平衡计算弱酸或弱碱溶液的pH。
\begin{exercise}
    已知乙酸$\ce{HAc}$($\ce{CH3COOH}$)的电离平衡常数$K_\mathrm{a}=1.76\times10^{-5}$,计算$0.2\ \mathrm{mol/L}$的乙酸溶液的pH
\end{exercise}
\begin{answer}
    $K_\mathrm{a}$的表达式为:
    \[K_\mathrm{a}=\dfrac{[\ce{H+}][\ce{Ac-}]}{[\ce{HAc}]}=1.76\times10^{-5}\]
    我们可以忽略水电离出的$\ce{H+}$,设$[\ce{H+}]=x$,则$[\ce{Ac-}]=[\ce{H+}]=x$。又因为$[\ce{Ac-}]+[\ce{HAc}]=0.2$,所以$[\ce{HAc}]=0.2-x$,代入$K_\mathrm{a}$的表达式中得:
    \[\dfrac{x^2}{0.2-x}=1.76\times10^{-5}\]
    直接解得:
    \[x=1.87\times10^{-3},\ \mathrm{pH}=2.73\]
\end{answer}\par
如果一元弱酸或者一元弱碱的$K_\mathrm{a}$或$K_\mathrm{b}$比较小,那么电离出的$[\ce{H+}]$或$[\ce{OH-}]$也会比较小(仍然比纯水中$10^{-7}$大),这样溶液中的$[\ce{A-}]$或$[\ce{B+}]$相比于$[\ce{HA}]$和$[\ce{BOH}]$也会比较小,可以认为$[\ce{HA}]=c$或者$[\ce{BOH}]=c$,这样计算起来会更方便,我们可以得到一个快速计算$[\ce{H+}]$或$[\ce{OH-}]$的公式。
\begin{formula}
    在浓度为$c$的弱酸溶液或弱碱溶液中,如果$K_\mathrm{a}$或$K_\mathrm{b}$较小:
    \[[\ce{H+}]=\sqrt{K_\mathrm{a}c}\]
    \[[\ce{OH-}]=\sqrt{K_\mathrm{b}c}\]
\end{formula}
\begin{derivation}
    \qquad 以$\ce{H+}$为例:弱酸$\ce{HA}$的电离平衡方程为:
    \[K_\mathrm{a}=\dfrac{[\ce{H+}][\ce{A-}]}{[\ce{HA}]}\]
    其中$[\ce{A-}]=[\ce{H+}]\approx0$,$[\ce{HA}]=c-[\ce{A-}]\approx c$,代入得:
    \[K_\mathrm{a}=\dfrac{[\ce{H+}]^2}{c}\]
    移项开方得:
    \[[\ce{H+}]=\sqrt{K_\mathrm{a}c}\]
\end{derivation}\par
多元弱酸与多元弱碱在水中的电离是分步的,每一步都有对应的平衡常数,第$n$步电离的电离平衡常数为$K_{\mathrm{a}n}$或$K_{\mathrm{b}n}$。\par
一般$K_{\mathrm{a/b}n}\gg K_{\mathrm{a/b}(n+1)}$,即第$n$步电离的程度远大于第$n+1$步。因此在计算多元弱酸或多元弱碱的pH时可以只考虑第一步电离。\par
对于多元弱酸或多元弱碱,有一个二级结论,即第二步电离出的离子浓度与$K_\mathrm{a2}$或$K_\mathrm{b2}$接近。
\begin{formula}
    对于多元弱酸$\ce{H_$n$A}$和多元弱碱$\ce{B(OH)_$n$}$:
    \[[\ce{H_$n-2$A^{2-}}]=K_\mathrm{a2}\]
    \[[\ce{B(OH)_$n-2$^{2+}}]=K_\mathrm{b2}\]
\end{formula}
\begin{derivation}
    \qquad 以多元弱酸$\ce{H_$n$A}$为例,只考虑第一步电离产生的氢离子,则:
    \[[\ce{H+}]=[\ce{H_$n-1$A^-}]\]
    第二步电离的平衡常数$K_\mathrm{a2}$的表达式为:
    \[K_\mathrm{a2}=\dfrac{[\ce{H+}][\ce{H_$n-2$A^{2-}}]}{[\ce{H_$n-1$A^-}]}\]
    由于$[\ce{H+}]=[\ce{H_$n-1$A^-}]$,上式可以约去这两项,得到:
    \[K_\mathrm{a2}=[\ce{H_$n-2$A^{2-}}]\]
\end{derivation}
\subsection{盐类的水解平衡}\label{2.8.3}
带有弱酸阴离子和弱碱阳离子的盐会发生水解反应,反应方程式如下:
\begin{reaction}
    盐类的水解:\par
    带有弱酸阴离子的盐$\ce{MA}$:
    \[\ce{A- + H2O <=> HA + OH-}\]
    带有弱碱阳离子的盐$\ce{BX}$:
    \[\ce{B+ + H2O <=> BOH + H+}\]
\end{reaction}\par
从方程式中可以看出,盐类的水解会改变溶液的pH。强碱弱酸盐显碱性,弱碱强酸盐显酸性,因此可以总结出“谁强显谁性”的判断法则。如果碰到弱酸弱碱盐,那么阴阳离子的两个水解反应会互相促进,被称为“双水解反应”。如果水解的程度很大,则可逆符号可以直接写成等号,可以认为发生了完全双水解反应。\par
盐类水解的平衡常数$K_\mathrm{h}$可以用水的离子积常数$K_\mathrm{w}$和弱酸弱碱的电离平衡常数$K_\mathrm{a}$或$K_\mathrm{b}$表示。
\begin{formula}
    带有弱酸阴离子的盐$\ce{MA}$:
    \[\ce{A- + H2O <=> HA + OH-}\]
    \[K_\mathrm{h}=\dfrac{K_\mathrm{w}}{K_\mathrm{a}}\]
    带有弱碱阳离子的盐$\ce{BX}$:
    \[\ce{B+ + H2O <=> BOH + H+}\]
    \[K_\mathrm{h}=\dfrac{K_\mathrm{w}}{K_\mathrm{b}}\]
\end{formula}\par
\begin{derivation}
    \qquad 以带有弱酸阴离子的盐$\ce{MA}$为例,反应$\ce{A- + H2O <=> HA + OH-}$的平衡常数为:
    \[K_\mathrm{h}=\dfrac{[\ce{HA}][\ce{OH-}]}{[\ce{A-}]}=\dfrac{[\ce{HA}][\ce{OH-}][\ce{H+}]}{[\ce{A-}][\ce{H+}]}\]
    由于$K_\mathrm{w}=[\ce{H+}][\ce{OH-}]$,$K_\mathrm{a}=\dfrac{[\ce{H+}][\ce{A-}]}{[\ce{HA}]}$,所以:
    \[K_\mathrm{h}=\dfrac{K_\mathrm{w}}{K_\mathrm{a}}\]
\end{derivation}\par
弱酸或弱碱与其对应的盐可以用来配制\textbf{缓冲溶液},将弱酸或弱碱与其对应的盐按接近1:1配制成混合溶液,就可以得到pH值接近$\mathrm{p}K_\mathrm{a}$的缓冲溶液,少量加酸或加碱,基本不会改变溶液的pH。
\subsection{配合物的配位平衡}\label{2.8.4}
向硫酸铜溶液中加入氨水,$\ce{Cu}$和$\ce{NH3}$会形成$\ce{[Cu(NH3)_4]^{2+}}$配离子。\par
在水中的配位化合物(简称配合物)也会有配位平衡,通常用稳定常数$K_\text{稳}$描述。
\begin{formula}
    对于配位反应:
    \[\ce{M + $n$L <=> ML_$n$}\]
    反应的稳定常数为:
    \[K_\text{稳}=\dfrac{[\ce{ML_$n$}]}{[\ce{M}][\ce{L}]^n}\]
    也可以分为多级稳定常数:
    \[K_{\text{稳}n}=\dfrac{[\ce{ML_$n$}]}{[\ce{ML_$n-1$}][\ce{L}]}\]
\end{formula}\par
稳定常数的数值决定了配合物形成的难易程度,稳定常数越大,配合物越稳定。
\subsection{沉淀溶解平衡}\label{2.8.5}
难溶的强电解质在水中的溶解也是可逆反应,我们可以用沉淀溶解平衡来描述,沉淀溶解平衡常数叫作溶度积常数,符号为$K_\mathrm{sp}$。
\begin{formula}
    沉淀溶解平衡:
    \[\ce{M_$m$N_$n$(s) <=> $m$M^{n+}(aq) + $n$N^{m-}(aq)}\]
    \[K_\mathrm{sp}=[\ce{M^{n+}}]^m[\ce{N^{n-}}]^n\]
\end{formula}\par
溶度积常数越小,难溶物的溶解度越小。溶度积常数$K_\mathrm{sp}$可以用于计算难溶电解质的溶解度$s$。
\begin{exercise}
    计算$25\ \mathrm{℃}$下列物质的溶解度:
    \begin{enumerate}
        \item $\ce{AgCl}$,$25\ \mathrm{℃}$下$K_\mathrm{sp}(\ce{AgCl})=1.8\times10^{-10}$
        \item $\ce{Ag2SO4}$,$25\ \mathrm{℃}$下$K_\mathrm{sp}(\ce{Ag2SO4})=1.2\times10^{-5}$
    \end{enumerate}
\end{exercise}
\begin{answer}
    \begin{enumerate}
        \item $K_\mathrm{sp}(\ce{AgCl})$的表达式为:
        \[K_\mathrm{sp}(\ce{AgCl})=[\ce{Ag+}][\ce{Cl-}]=1.8\times10^{-10}\]
        又因为在反应$\ce{AgCl(s) <=> Ag+(aq) + Cl-(aq)}$中$[\ce{Ag+}]=[\ce{Cl-}]=\dfrac{s}{c^0}$,所以:
        \[\left(\dfrac{s}{c^0}\right)^2=1.8\times10^{-10}\]
        解得$s=1.34\times10^{-5}\ \mathrm{mol/L}$
        \item $K_\mathrm{sp}(\ce{Ag2SO4})$的表达式为:
        \[K_\mathrm{sp}(\ce{Ag2SO4})=[\ce{Ag+}]^2[\ce{SO4^{2-}}]\]
        又因为在反应$\ce{Ag2SO4(s) <=> 2Ag+(aq) + SO4^{2-}(aq)}$中$[\ce{Ag+}]=2[\ce{SO4^{2-}}]=\dfrac{2s}{c^0}$,所以:
        \[\left(\dfrac{2s}{c^0}\right)^2\left(\dfrac{s}{c^0}\right)=1.2\times10^{-5}\]
        解得$s=0.014\ \mathrm{mol/L}$
    \end{enumerate}
\end{answer}\par
如果求得的离子浓度小于$10^{-5}\ \mathrm{mol/L}$,可以认为离子已经完全除去。
\subsection{水溶液中平衡的移动}\label{2.8.6}
水溶液中的各类平衡很容易受到其他物质的影响。比如在$\ce{HAc}$溶液中加入$\ce{HCl}$或$\ce{NaAc}$,会使$\ce{HAc}$的电离平衡逆向移动;在含有$\ce{Ba^{2+}}$的溶液中加入大量$\ce{Na2SO4}$,可以使$\ce{Ba^{2+}}$的浓度大大减小。\par
我们需要学会使用平衡常数计算平衡移动后溶液中某些离子的浓度。
\begin{exercise}
    回答下列问题:
    \begin{enumerate}
        \item 已知$K_\mathrm{a}(\ce{HAc})=1.76\times10^{-5}$,求含有$0.1\ \mathrm{mol/L}\ \ce{HAc}$与$0.1\ \mathrm{mol/L}\ \ce{HCl}$的溶液中$\ce{Ac-}$的浓度
        \item 向$1\ \mathrm{L}\ 0.1\ \mathrm{mol/L}\ \ce{BaCl2}$溶液中加入$0.3\ \mathrm{mol}\ \ce{Na2SO4}$固体,已知$K_\mathrm{sp}(\ce{BaSO4})=1.1\times10^{-10}$求此时溶液中$\ce{Ba^{2+}}$的浓度
    \end{enumerate}
\end{exercise}
\begin{answer}
    \begin{enumerate}
        \item 设$[\ce{Ac-}]=x$,则$[\ce{H+}]=0.1+x$,$[\ce{HAc}]=0.1-x$,$\ce{HAc}$的电离平衡方程为:
        \[K_\mathrm{a}(\ce{HAc})=\dfrac{[\ce{H+}][\ce{Ac-}]}{[\ce{HAc}]}=\dfrac{(0.1+x)x}{0.1-x}=1.76\times10^{-5}\]
        解得$x=1.76\times10^{-5}$,即$c(\ce{Ac-})=1.76\times10^{-5}\ \mathrm{mol/L}$。\par
        我们发现$c(\ce{Ac-})\approx K_\mathrm{a}$,这是因为在电离平衡方程中$0.1+x\approx0.1\approx0.1-x$,即$[\ce{H+}]\approx[\ce{HAc}]$,这样的结果是合理的。
        \item 溶液中发生反应:
        \[\ce{Ba^{2+}(aq) + SO4^{2-}(aq) <=> BaSO4(s)}\]
        从$K_\mathrm{sp}(\ce{BaSO4})$可以得知$\ce{BaSO4}$溶解度较小,$[\ce{Ba^{2+}}]\approx0$,因此$\Delta[\ce{Ba^{2+}}]\approx0.1$,$[\ce{SO4^{2-}}]=\dfrac{0.3\ \mathrm{mol}}{1\ \mathrm{L}\times c^0}-\Delta[\ce{Ba^{2+}}]\approx0.2$。$\ce{BaSO4}$的沉淀溶解平衡方程为:
        \[K_\mathrm{sp}(\ce{BaSO4})=[\ce{Ba^{2+}}][\ce{SO4^{2-}}]=1.1\times10^{-10}\]
        即:
        \[[\ce{Ba^{2+}}]\times0.2=1.1\times10^{-10}\]
        解得$[\ce{Ba^{2+}}]=5.5\times10^{-10}$,即$c(\ce{Ba^{2+}})=5.5\times10^{-10}\ \mathrm{mol/L}$。
    \end{enumerate}
\end{answer}

\end{document}