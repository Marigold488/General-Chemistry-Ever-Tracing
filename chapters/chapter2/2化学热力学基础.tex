\documentclass[../../GCET-main.tex]{subfiles}

\begin{document}

\setcounter{chapter}{1}
\chapter{化学热力学基础}\label{2}
\textbf{化学热力学}是用热力学的基本规律研究化学反应及相变中的能量、方向、限度的科学。在\autoref{1}中我们已经较为全面地分析了相变,但对于其中相变焓的定义、\autoref{formula1.8}克劳修斯-克拉贝龙方程、\autoref{formula1.9}克拉贝龙方程的来历,我们还没有进一步追寻。\par
本章我们会学习热力学四大定律和一系列\textbf{状态函数}及\textbf{过程量(途径函数)},并探索这些热力学的基础知识如何应用于化学反应。
\subfile{section1/2.1热力学基础.tex}
\subfile{section2/2.2热力学第一定律.tex}
\subfile{section3/2.3焓变的计算.tex}
\subfile{section4/2.4热力学第二定律.tex}
\subfile{section5/2.5熵变的计算.tex}
\subfile{section6/2.6吉布斯自由能.tex}
\subfile{section7/2.7化学平衡.tex}
\subfile{section8/2.8水溶液中的化学平衡.tex}
\subfile{section9/2.9pVT和相变计算专题.tex}
\phantomsection
\addcontentsline{toc}{section}{本章小结}
\section*{本章小结}
本章我们学习了基础的化学热力学,了解了热力学四大定律和内能、功、热、焓、熵、吉布斯自由能等物理量,掌握了理想气体的各种过程中相关物理量的计算。更加重要的是,我们学习了化学反应中的焓变、熵变、吉布斯自由能变和化学平衡,将热力学的知识用到了化学中,用热力学的方法判断化学反应的自发性和限度。\par
本章的推导过程较多,补充了普通化学教材中完全没有的推导过程,很多地方参考了物理化学教材。虽然这些内容难度比较大,但掌握这些推导过程以及最后的专题会对解热力学计算题有很大帮助,希望能帮助大家搭建课本到考试的桥梁。\par
本章的知识是整个普通化学的主干,第一章和第三章的知识都以这一章的知识为基础,足见本章的重要性。在后续物理课和物理化学课我们将再次接触热力学,希望这里的知识到那时候仍然能帮助读者。
\subfile{exercise/2exercise.tex}
\subfile{exercise/2answer.tex}

\end{document}