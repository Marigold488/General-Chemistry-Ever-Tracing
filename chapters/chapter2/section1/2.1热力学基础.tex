\documentclass[../../../GCET-main.tex]{subfiles}

\begin{document}

\section{热力学基础}\label{2.1}
在正式学习化学热力学之前,我们有必要明确一下热力学的研究对象和基本术语。\par
\textbf{热力学的研究对象}是含有大量粒子的宏观系统,它的原理、结论不能用于描述单个或少数几个微观粒子。所以在上一章的习题中,第2题是明显错误的。一个气体分子无法满足理想气体状态方程,它根本不是热力学的研究对象。

\subsection{体系与环境}\label{2.1.1}
热力学把世界分为\textbf{体系(系统)}\footnote{两者含义完全等价,本书中均会出现,所有定义中带括号的名词与此同理}和\textbf{环境},体系是热力学研究的部分,环境是对体系有影响的剩余部分。例如,要研究烧杯中的化学反应,我们可以把烧杯中的所有物质看成体系,把烧杯本身和烧杯外面的整个世界——包括你自己看成环境。
\begin{definition}
    \textbf{体系(系统)}:作为研究对象的一部分物质或空间\par
    \textbf{环境}:体系之外与体系密切相关、能够产生相互作用的其余一切物质或空间
\end{definition}\par
根据体系和环境之间是否存在物质交换与能量交换,我们可以把体系分成三类:\textbf{孤立体系(隔离体系)}、\textbf{封闭体系}、\textbf{开放体系(敞开体系)}。
\begin{definition}
    \textbf{孤立体系(隔离体系)}:既没有物质交换,也没有能量交换的体系\par
    \textbf{封闭体系}:没有物质交换,但有能量交换的体系\par
    \textbf{开放体系(敞开体系)}:既有物质交换,又有能量交换的体系
\end{definition}\par
读者可能会有这样的疑问:按照是否存在物质交换和是否存在能量交换分,应该可以分出四类体系,为什么没有“有物质交换但是没有能量交换”的体系呢?这是因为物质必定携带着能量,\textbf{有物质交换必然有能量交换}。\par
关于体系,我们还有以下结论:第一,如果把体系和环境看成一个新的体系,那么这个体系是一个孤立体系,\textbf{整个宇宙是一个孤立体系};第二,热力学中最常用的体系是封闭体系,我们只需要关注封闭体系的能量而不用关心物质交换。
\subsection{状态与状态函数}\label{2.1.2}
\subsubsection{状态、状态函数、状态方程的定义}
在\autoref{1.1}中我们已经提到过\textbf{状态}、\textbf{状态函数}、\textbf{状态方程},在这里我们再次给出这三个名词的定义。
\begin{definition}
    \textbf{状态}:系统所有的热力学性质的综合表现\par
    \textbf{状态函数}:仅由系统的状态决定,而与系统达到该状态的变化途径无关的物理量\par
    \textbf{状态方程}:描述处于平衡态的热力学系统中,各宏观状态函数之间定量关系的数学表达式
\end{definition}\par
系统的状态由状态函数描述,如果描述系统的所有物理量(状态函数)确定,那么系统的状态也就确定。系统的状态确定,则系统的热力学性质也确定,这些热力学性质被称为状态函数。状态和状态函数是一对互相依存的定义。\par
我们在第一章中使用过的温度$T$、体积$V$、压强$p$,碰到过的焓$H$,以及我们这章即将学习的内能$U$、熵$S$、吉布斯自由能$G$、亥姆霍兹自由能$A$都是状态函数。这八个状态函数是整个热力学的基础。此外,物质的量$n$也是状态函数,但在封闭体系中物质的量固定不变,只有在开放体系中$n$才会对体系有影响。\par
状态函数只与状态有关,所以在一个过程中状态函数的变化只与始末状态有关,而与途径无关,这区别于下一小节中的\textbf{过程量(途径函数)}。系统状态的微小变化引起的状态函数$X$的微小变化可以用全微分$\diff X$表示;状态函数$X$的较大变化用$\Delta X$表示。\par
系统的许多性质间有一定联系,在没有外场作用、物质的量及组成确定的均相系统中,只需要指定两个可以独立变化的性质,系统的状态就可以确定。也就是说,只需要指定两个状态函数作为变量,其余的状态函数都可以表示成这两个状态函数的二元函数。状态函数之间的关系可以用状态方程描述,比如理想气体的$n$确定后,$T$、$V$、$p$三个状态函数中只需要确定其中两个,就可以用\autoref{formula1.3}理想气体状态方程计算出剩下的状态函数。
\subsubsection{状态函数的分类}
按照热力学系统的状态函数的数值是否与物质的数量有关,可以将状态函数分为\textbf{广度量(广度性质)}和\textbf{强度量(强度性质)}。
\begin{definition}
    \textbf{广度量(广度性质)}:与物质的数量成正比的性质,具有加和性\par
    \textbf{强度量(强度性质)}:与物质的数量无关的性质,不具有加和性
\end{definition}\par
热力学常见的物理量中$p$和$T$是强度量,$n$、$V$等是广度量。广度量的加和性指的是热力学系统中系统的广度量是各部分的广度量之和,而强度量在系统各部分相等。比如混合两杯水,得到的水的体积和物质的量是原来两杯水之和,而温度却不是。\par
读者可能会有这样的疑问:为什么压强$p$是强度量,但在\autoref{formula1.5}中我们却认为混合的理想气体的总压是各部分分压之和,这是否违背了强度量不具有加和性?答案当然是否定的。这里的相加与系统中各部分广度量的相加不同,系统的压强可以看作由系统中的不同组分贡献,这些组分不能看作独立系统,而是共存于同一个均匀的系统中,系统的各个部分的压强等于系统压强,因此在计算总压时可以按照贡献叠加。\par
广度量和强度量有一些运算性质:广度量除以广度量等于强度量,如摩尔体积$V_\mathrm{m}=\dfrac{V}{n}$是强度量;强度量乘以广度量是广度量。
\subsubsection{平衡态}
一般我们研究的热力学状态主要是\textbf{平衡态},研究的变化也基本是从平衡态到平衡态的变化。平衡态的定义如下。
\begin{definition}
    \textbf{平衡态}:在一定条件下,系统各个相的热力学性质不随时间变化,且将系统与环境隔离后,系统的性质仍不改变的状态
\end{definition}\par
系统处于平衡态需要满足以下三个条件:\textbf{热平衡}、\textbf{力平衡}和\textbf{物质平衡},其中\textbf{物质平衡}还可以细分为相平衡和化学平衡。
\begin{definition}
    \textbf{热平衡}:系统有单一的温度\par
    \textbf{力平衡}:系统有单一的压强\par
    \textbf{物质平衡}:系统内部各个相中的各个组分的物质的量不随时间改变
\end{definition}\par
总而言之,系统的温度、压强、各个相各个组分的物质的量均不随时间改变时,系统处于平衡态,这几个条件缺一不可。
\subsection{过程、途径与过程量}\label{2.1.3}
\subsubsection{过程与途径}
当系统从一个状态(\textbf{始态})变化到另一个状态(\textbf{终态})时,系统就进行了一个\textbf{过程};系统从\textbf{始态}变化到\textbf{终态}可以有多种方式,这些方式叫作\textbf{途径}。
\begin{definition}
    \textbf{始态}:系统发生变化之前所处的热力学状态\par
    \textbf{终态}:系统变化结束之后所处的热力学状态\par
    \textbf{过程}:在外界条件改变时,系统从始态到达终态所经历的状态变化\par
    \textbf{途径}:系统从始态变化到终态所经历的具体方式与步骤
\end{definition}\par
这里的过程和途径类似物理学中的位移和路程,过程只与始态和终态有关,而途径是始态到达终态具体的路径。同一个过程可以对应多种途径。\par
根据过程进行的特定条件,我们可以把过程分为\textbf{恒温过程}、\textbf{恒压过程}、\textbf{恒容过程}、\textbf{绝热过程}、\textbf{循环过程},这些过程的定义如下。
\begin{definition}
    \textbf{恒温过程}:系统温度与环境温度相等且不变的过程\par
    \textbf{恒压过程}:系统压强与环境压强相等且不变的过程\par
    \textbf{恒容过程}:系统体积不变的过程\par
    \textbf{绝热过程}:系统与环境间无热交换的过程\par
    \textbf{循环过程}:系统的始态与终态相同的过程
\end{definition}\par
根据循环过程的定义和\autoref{definition2.3}中状态函数的性质,循环过程的始态和终态相同,状态函数也相同,因此循环过程中状态函数的变化$\Delta X=0$。\par
根据过程的变化类型分,我们可以把过程分为\textbf{单纯$pVT$变化过程}、\textbf{相变化过程}、\textbf{化学变化过程}等。
\begin{definition}
    \textbf{单纯$pVT$变化过程}:只有压力、体积、温度变化,无相变、无化学反应的过程\par
    \textbf{相变化过程}:发生相变的过程\par
    \textbf{化学变化过程}:发生化学反应的过程
\end{definition}\par
根据过程是否可逆,我们可以把过程分为\textbf{可逆过程}和\textbf{不可逆过程}。
\begin{definition}
    \textbf{可逆过程}:系统从始态变化到终态的过程中系统每一瞬间都无限接近平衡态,并可以通过无限小的改变使系统和环境完全复原的过程\par
    \textbf{不可逆过程}:系统从始态变化到终态的过程中系统并非每一瞬间都处于平衡态,无论采用何种方法都无法使系统与环境同时完全复原的过程
\end{definition}\par
可逆过程是一种理想过程,它没有任何推动力,现实中并不存在这样的实际过程,只能找到无限接近可逆过程的实际过程,\textbf{实际过程都是不可逆的}。\par
根据过程的自发性,我们可以把过程分为\textbf{自发过程}和\textbf{不自发过程}。
\begin{definition}
    \textbf{自发过程}:在一定条件下,不需要外界持续做功,就能自动进行的过程\par
    \textbf{不自发过程}:在没有外界对系统持续做功的条件下,不能自动进行的过程
\end{definition}\par
自发过程一定是不可逆过程。\textbf{可逆过程既不是自发过程也不是不自发过程},因为它是无推动力的理想极限过程,无自发进行的趋势,也无需外界持续做功。\par
这几种过程既可以作为独立的过程存在,也可以作为其他过程的途径。一个过程可以细分为几个过程,比如从始态$(p_1,V_1,T)$变化到终态$(p_2,V_2,T)$的过程,可以只经过一个恒温可逆过程,也可以拆分为一个恒压可逆过程和一个恒容可逆过程进行,在这里这几种过程可以用来描述另一个过程的途径。
\subsubsection{过程量(途径函数)}
\autoref{definition2.3}中的状态函数只由系统的状态决定,其变化只与系统的始态和终态有关,而与途径无关。而\textbf{过程量(途径函数)}不仅与系统的始态和终态有关,还和具体路径、过程有关。
\begin{definition}
    \textbf{过程量(途径函数)}:描述系统从一个状态到另一个状态所经历的过程特征(具体途径)的物理量(函数)
\end{definition}\par
热力学中常见的两个过程量是功和热,符号分别为$W$和$Q$。\par
过程量不是状态函数,所以我们不能说某个系统有多少过程量,只能说某个过程中改变了多少过程量。系统在无限小过程中传递的过程量的微小量,可以用$\delta$表示,比如$\delta W$和$\delta Q$;过程量的较大变化就是过程量本身,如$Q$和$W$,不能在前面加$\Delta$。
\subsection{上标与下标}\label{2.1.4}
化学热力学中会出现很多上标与下标,主要分为以下几种:
\begin{enumerate}
    \item 表示对象的下标(一般加在物理量右边):
    \begin{enumerate}
        \item 系统:sys(system)
        \item 环境:sur(surroundings)
        \item 外(环境):ex(external)
        \item 总:tot(total)
    \end{enumerate}
    \item 表示摩尔量的下标(一般加在物理量右边):m(molar)
    \item 表示过程条件的下标(一般加在物理量右边):
    \begin{enumerate}
        \item 恒温:T
        \item 恒压:p
        \item 恒容:V
        \item 绝热:ad(adiabatic)
        \item 可逆:rev(reversible)
        \item 不可逆:irr(irreversible)
    \end{enumerate}
    \item 表示过程类型的下标(没有其他下标时加在物理量右边,有其他下标时加在物理量左边):
    \begin{enumerate}
        \item 反应:r(reaction)
        \item 生成:f(formation)
        \item 燃烧:c(combustion)
        \item 相变/晶型转变:trs(transition)
        \item 汽化:vap(vaporization)
        \item 熔化:fus(fusion)
        \item 升华:sub(sublimation)
        \item 混合:mix(mixing)
        \item 溶解:sol(solution)
        \item 活化:a(activation)
    \end{enumerate}
    \item 表示标准状态的上标(一般加在物理量右边):0、$\circ$、$\theta$,本书中统一用0
    \item 表示平衡状态的上标(一般加在物理量右边):eq
\end{enumerate}
\subsection{热力学第零定律}\label{2.1.5}
\textbf{热力学第零定律}也被称为\textbf{热平衡定律},它的表述如下。
\begin{formula}
    \textbf{热力学第零定律}:如果两个系统分别与第三个系统处于热平衡,则这两个系统彼此也必处于热平衡。\par
    若设第一个系统为$A$,第二个系统为$B$,第三个系统为$C$,则热力学第零定律可以表示为:
    \[T_A=T_C,\ T_B=T_C\Longrightarrow T_A=T_B\]
\end{formula}\par
热力学第零定律给出了温度的科学定义,是热力学的基础,但是人们在发现了热力学第一定律、第二定律、第三定律之后才意识到这个基础,因此热平衡排在最前面,被称为热力学第零定律。

\end{document}