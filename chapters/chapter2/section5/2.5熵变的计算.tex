\documentclass[../../../GCET-main.tex]{subfiles}

\begin{document}

\section{熵变的计算***}\label{2.5}
\subsection{单纯$pVT$变化过程的熵变}\label{2.5.1}
从熵的定义来看,熵的微分等于可逆过程的热温商。对于可逆过程,过程中的熵变可以直接用定义式$\diff S=\dfrac{\delta Q}{T}$积分计算;对于不可逆过程,过程中的熵变不能直接用定义式计算,需要\textbf{利用熵是状态函数的性质},只要找到一个始态和终态都相同的可逆过程,再用可逆过程的热计算出熵变即可。\par
\subsubsection{可逆过程熵变}
可逆过程熵变的计算与热的计算息息相关,我们先给出公式,随后一一推导。
\begin{formula}
    从状态1$(p_1,V_1,T_1)$到状态2$(p_2,V_2,T_2)$可逆过程熵变的计算:\par
    \textbf{绝热可逆过程}:
    \[\diff S=0\]
    \[\Delta S=0\]
    \textbf{恒温可逆过程}:
    \[\diff S=\dfrac{\delta Q_\mathrm{rev}}{T}=\dfrac{-\delta W_\mathrm{rev}}{T}\]
    \[\Delta S=\dfrac{Q_\mathrm{rev}}{T}=\dfrac{-W_\mathrm{rev}}{T}=nR\ln\dfrac{V_2}{V_1}\]
    第二个式子中第三个等号要求理想气体\par
    \textbf{恒容可逆过程}:
    \[\diff S=\dfrac{C_\mathrm{V}\diff T}{T}\]
    如果$C_\mathrm{V}$视为常数:
    \[\Delta S=C_\mathrm{V}\ln\dfrac{T_2}{T_1}\]
    \textbf{恒压可逆过程}:
    \[\diff S=\dfrac{C_\mathrm{p}\diff T}{T}\]
    如果$C_\mathrm{p}$视为常数:
    \[\Delta S=C_\mathrm{p}\ln\dfrac{T_2}{T_1}\]
    $C_\mathrm{V}$和$C_\mathrm{p}$视为常数,对于理想气体的\textbf{任意可逆过程}均成立的三个公式:\par
    $T,V$:
    \[\Delta S=C_\mathrm{V}\ln\dfrac{T_2}{T_1}+nR\ln\dfrac{V_2}{V_1}\]
    $T,p$:
    \[\Delta S=C_\mathrm{p}\ln\dfrac{T_2}{T_1}-nR\ln\dfrac{p_2}{p_1}\]
    $V,p$:
    \[\Delta S=C_\mathrm{p}\ln\dfrac{V_2}{V_1}+C_\mathrm{V}\ln\dfrac{p_2}{p_1}\]
\end{formula}
\begin{derivation}
    \textbf{绝热可逆过程}:\par
    \qquad 根据\autoref{definition2.19}熵的定义:
    \[\diff S=\dfrac{\delta Q_\mathrm{rev}}{T}\]
    绝热过程$\delta Q_\mathrm{rev}=0$,所以:
    \[\diff S=0\]
    积分得:
    \[\Delta S=0\]
\end{derivation}
\begin{derivation}
    \textbf{恒温可逆过程}:\par
    \qquad 根据\autoref{definition2.19}熵的定义:
    \[\diff S=\dfrac{\delta Q_\mathrm{rev}}{T}\]
    积分得:
    \[\Delta S=\int_{1}^{2}\dfrac{\delta Q_\mathrm{rev}}{T}\]
    恒温过程$T$是常数,可提出积分外,得到:
    \[\Delta S=\dfrac{1}{T}\int_{1}^{2}\delta Q_\mathrm{rev}=\dfrac{Q_\mathrm{rev}}{T}\]
    根据\autoref{formula2.3}热力学第一定律:
    \[\Delta U=Q_\mathrm{rev}+W_\mathrm{rev}=0\Longrightarrow Q_\mathrm{rev}=-W_\mathrm{rev}\]
    代入得:
    \[\Delta S=\dfrac{-W_\mathrm{rev}}{T}\]
    根据\autoref{formula2.7},对于理想气体有:
    \[W_\mathrm{rev}=-nRT\ln\dfrac{V_2}{V_1}\]
    代入得:
    \[\Delta S=nR\ln\dfrac{V_2}{V_1}\]
\end{derivation}
\begin{derivation}
    \textbf{恒容可逆过程}:\par
    \qquad 根据\autoref{definition2.19}熵的定义:
    \[\diff S=\dfrac{\delta Q_\mathrm{rev}}{T}\]
    根据\autoref{formula2.11}恒容热公式,$\delta Q_\mathrm{rev}=C_\mathrm{V}\diff T$,代入得:
    \[\diff S=\dfrac{C_\mathrm{V}\diff T}{T}\]
    两边积分得:
    \[\Delta S=\int_{T_1}^{T_2}\dfrac{C_\mathrm{V}\diff T}{T}=C_\mathrm{V}\ln T\bigg|_{T_1}^{T_2}=C_\mathrm{V}\ln\dfrac{T_2}{T_1}\]
\end{derivation}
\begin{derivation}
    \textbf{恒压可逆过程}:\par
    \qquad 根据\autoref{definition2.19}熵的定义:
    \[\diff S=\dfrac{\delta Q_\mathrm{rev}}{T}\]
    根据\autoref{formula2.11}恒压热公式,$\delta Q_\mathrm{rev}=C_\mathrm{p}\diff T$,代入得:
    \[\diff S=\dfrac{C_\mathrm{p}\diff T}{T}\]
    两边积分得:
    \[\Delta S=\int_{T_1}^{T_2}\dfrac{C_\mathrm{p}\diff T}{T}=C_\mathrm{p}\ln T\bigg|_{T_1}^{T_2}=C_\mathrm{p}\ln\dfrac{T_2}{T_1}\]
\end{derivation}\par
\begin{derivation}
    \textbf{任意可逆过程}:\par
    $T,V$:\par
    \qquad 根据\autoref{definition2.19}熵的定义:
    \[\diff S=\dfrac{\delta Q_\mathrm{rev}}{T}\]
    根据\autoref{formula2.3}热力学第一定律:
    \[\diff U=\delta Q_\mathrm{rev}+\delta W_\mathrm{rev}=\delta Q_\mathrm{rev}-p_\mathrm{ex}\diff V\Longrightarrow \delta Q_\mathrm{rev}=\diff U+p_\mathrm{ex}\diff V\]
    可逆过程$p_\mathrm{ex}=p$,所以:
    \[\delta Q_\mathrm{rev}=\diff U+p\diff V\]
    代回熵的计算式得:
    \[\diff S=\dfrac{\diff U+p\diff V}{T}\]
    根据\autoref{formula2.12},$\diff U=C_\mathrm{V}\diff T$,代入得:
    \[\diff S=\dfrac{C_\mathrm{V}\diff T+p\diff V}{T}\]
    根据\autoref{formula1.3}理想气体状态方程,$pV=nRT\Longrightarrow p=\dfrac{nRT}{V}$,代入得:
    \[\diff S=\dfrac{C_\mathrm{V}\diff T}{T}+\dfrac{nR\diff V}{V}\]
    积分得:
    \[\Delta S=C_\mathrm{V}\ln\dfrac{T_2}{T_1}+nR\ln\dfrac{V_2}{V_1}\]
    $T,p$:\par
    \qquad 根据\autoref{definition2.19}熵的定义:
    \[\diff S=\dfrac{\delta Q_\mathrm{rev}}{T}\]
    由上面的推导过程:
    \[\delta Q_\mathrm{rev}=\diff U+p\diff V\]
    根据\autoref{definition2.15}焓的定义:
    \[H=U+pV\Longrightarrow\diff H=\diff U+p\diff V+V\diff p\Longrightarrow\diff U+p\diff V=\diff H-V\diff p\]
    代入$\delta Q_\mathrm{rev}$的表达式得:
    \[\delta Q_\mathrm{rev}=\diff H-V\diff p\]
    代回熵的计算式得:
    \[\diff S=\dfrac{\diff H-V\diff p}{T}\]
    根据\autoref{formula2.12},$\diff H=C_\mathrm{p}\diff T$,代入得:
    \[\diff S=\dfrac{C_\mathrm{p}\diff T-V\diff p}{T}\]
    根据\autoref{formula1.3}理想气体状态方程,$pV=nRT\Longrightarrow V=\dfrac{nRT}{p}$,代入得:
    \[\diff S=\dfrac{C_\mathrm{p}\diff T}{T}-\dfrac{nR\diff p}{p}\]
    积分得:
    \[\Delta S=C_\mathrm{p}\ln\dfrac{T_2}{T_1}-nR\ln\dfrac{p_2}{p_1}\]
    $V,p$:
    根据\autoref{formula1.3}理想气体状态方程:
    \[pV=nRT\Longrightarrow\dfrac{pV}{T}=nR\Longrightarrow\dfrac{p_1V_1}{T_1}=\dfrac{p_2V_2}{T_2}=nR\]
    把$T_1$和$T_2$移到一边得:
    \[\dfrac{T_2}{T_1}=\dfrac{p_2}{p_1}\times\dfrac{V_2}{V_1}\]
    把这个式子代入刚刚得到的$T,p$的公式中:
    \begin{align*}
        \Delta S&=C_\mathrm{p}\ln\left(\dfrac{p_2}{p_1}\times\dfrac{V_2}{V_1}\right)-nR\ln\dfrac{p_2}{p_1}=C_\mathrm{p}\ln\dfrac{p_2}{p_1}+C_\mathrm{p}\ln\dfrac{V_2}{V_1}-nR\ln\dfrac{p_2}{p_1}\\
        &=C_\mathrm{p}\ln\dfrac{V_2}{V_1}+(C_\mathrm{p}-nR)\ln\dfrac{p_2}{p_1}
    \end{align*}
    根据\autoref{formula2.13}理想气体恒容热容与恒压热容的关系,$C_\mathrm{p}=C_\mathrm{V}+nR\Longrightarrow C_\mathrm{p}-nR=C_\mathrm{V}$,代入得:
    \[\Delta S=C_\mathrm{p}\ln\dfrac{V_2}{V_1}+C_\mathrm{V}\ln\dfrac{p_2}{p_1}\]
\end{derivation}\par
\subsubsection{不可逆过程熵变}
下面展示一个计算不可逆过程熵变的例子,方法是先找到始态和终态相同的可逆过程,再利用可逆过程的公式计算。
\textbf{自由膨胀}是气体在绝热容器中向真空膨胀,是不可逆过程。理想气体自由膨胀的熵变公式如下。
\begin{formula}
    理想气体自由膨胀过程的熵变:
    \[\Delta S=nR\ln\dfrac{V_2}{V_1}\]
\end{formula}
我们先分析理想气体自由膨胀的特征:
\begin{enumerate}
    \item $p_\mathrm{ex}=0$,$W=-p_\mathrm{ex}\Delta V=0$
    \item 绝热容器中$Q=0$
    \item 根据\autoref{formula2.3}热力学第一定律,$\Delta U=Q+W=0$
    \item 因为$\Delta U=Q+W=0$,所以$T$不变
\end{enumerate}\par
所以我们可以计算恒温可逆膨胀的熵变来间接计算自由膨胀的熵变,推导过程如下。
\begin{derivation}
    \qquad 从状态1$(p_1,V_1,T)$到状态2$(p_2,V_2,T)$的自由膨胀过程熵变等于始态和终态相同的恒温膨胀过程的熵变。
    根据\autoref{formula2.24},恒温可逆过程的熵变为:
    \[\Delta S=\dfrac{Q_\mathrm{rev}}{T}=\dfrac{-W_\mathrm{rev}}{T}=nR\ln\dfrac{V_2}{V_1}\]
    理想气体自由膨胀只能用最后一个式子。
\end{derivation}\par
理想气体自由膨胀的$Q=0$,但是$\Delta S\neq 0$,因为自由膨胀过程不可逆,不能用不可逆过程的热计算熵变。
\subsection{混合过程的熵变}\label{2.5.2}
不同理想气体的混合过程熵变为各部分理想气体的熵变之和。做题时分别计算各部分理想气体的熵变再求和即可。
\begin{exercise}
    一容器中有一隔板,将$1\ \mathrm{mol}\ \ce{N2(g)}$和$1\ \mathrm{mol}\ \ce{O2(g)}$隔开,体系和环境温度恒定,均为$300\ \mathrm{K}$,体积都是$1\ \mathrm{L}$。$\ce{N2(g)}$和$\ce{O2(g)}$都可以视为理想气体。求:
    \begin{enumerate}
        \item 抽掉隔板之后混合过程的熵变$\Delta_\mathrm{mix}S_1$;
        \item 再将容器压缩至总体积为$1\ \mathrm{L}$,求此过程的熵变$\Delta S_2$;
        \item 求上述两过程的熵变之和$\Delta S_3$;
        \item 将$\ce{O2(g)}$换成$\ce{N2(g)}$,求抽掉隔板之后的熵变$\Delta S_4$。
    \end{enumerate}
\end{exercise}
\begin{answer}
    \begin{enumerate}
        \item $\ce{N2(g)}$和$\ce{O2(g)}$都进行自由膨胀,根据\autoref{formula2.25}:
        \[\Delta S(\ce{N2})=nR\ln\dfrac{V_2}{V_1}=1\ \mathrm{mol}\times8.314\ \mathrm{J/(mol\cdot K)}\times\ln\dfrac{2\ \mathrm{L}}{1\ \mathrm{L}}=8.314\ln 2\ \mathrm{J/K}\]
        \[\Delta S(\ce{O2})=nR\ln\dfrac{V_2}{V_1}=1\ \mathrm{mol}\times8.314\ \mathrm{J/(mol\cdot K)}\times\ln\dfrac{2\ \mathrm{L}}{1\ \mathrm{L}}=8.314\ln 2\ \mathrm{J/K}\]
        \[\Delta_\mathrm{mix}S_1=\Delta S(\ce{N2})+\Delta S(\ce{O2})=11.53\ \mathrm{J/K}\]
        \item 无论压缩是否可逆,都可以当成理想气体恒温可逆压缩过程计算,根据\autoref{formula2.24}:
        \[\Delta S_2=nR\ln\dfrac{V_2}{V_1}=2\ \mathrm{mol}\times 8.314\ \mathrm{J/(mol\cdot K)}\times\ln\dfrac{1\ \mathrm{L}}{2\ \mathrm{L}}=-11.53\ \mathrm{J/K}\]
        \item 相加即可:
        \[\Delta S_3=\Delta_\mathrm{mix}S_1+\Delta S_2=16.628\ln 2\ \mathrm{J/K}-16.628\ln 2\ \mathrm{J/K}=0\]
        \item 抽掉隔板后两部分气体种类和状态完全相同,系统不会发生任何变化,$\Delta S=0$。
    \end{enumerate}
\end{answer}
\subsection{可逆相变过程的熵变}\label{2.5.3}
\textbf{可逆相变过程}\footnote{即在相边界进行的相变}的温度和压强均为定值,摩尔相变熵的计算公式如下。
\begin{formula}
    相变过程的摩尔熵变:
    \[\Delta_\mathrm{trs}S_\mathrm{m}=\dfrac{\Delta_\mathrm{trs}H_\mathrm{m}}{T_\mathrm{trs}}\]
\end{formula}\par
这里没有规定标准状态,因为不同压强下的摩尔相变熵都可以这么计算,如果压强是$p^0$,那可以在公式的焓变和熵变中加上上标$0$。此公式的推导过程如下。
\begin{derivation}
    \qquad 根据\autoref{definition2.19}熵的定义:
    \[\diff_\mathrm{trs}S_\mathrm{m}=\dfrac{\delta Q_\mathrm{rev}}{T_\mathrm{trs}}\]
    积分得:
    \[\Delta_\mathrm{trs}S_\mathrm{m}=\int_{1}^{2}\dfrac{\delta Q_\mathrm{rev}}{T_\mathrm{trs}}\]
    $T_\mathrm{trs}$是常数,可提出积分外,得到:
    \[\Delta_\mathrm{trs}S_\mathrm{m}=\dfrac{1}{T_\mathrm{trs}}\int_{1}^{2}\delta Q_\mathrm{rev}=\dfrac{Q_\mathrm{rev}}{T_\mathrm{trs}}\]
    $p$也是常数,根据\autoref{formula2.9}恒压热等于焓变,$Q_\mathrm{rev}=\Delta_\mathrm{trs}H_\mathrm{m}$,代入得:
    \[\Delta_\mathrm{trs}S_\mathrm{m}=\dfrac{\Delta_\mathrm{trs}H_\mathrm{m}}{T_\mathrm{trs}}\]
\end{derivation}\par
一般可以查到物质的标准摩尔相变熵$\Delta_\mathrm{trs}S_\mathrm{m}^0$,需要计算含相变的过程的熵时可以在标准状况下计算。
\subsection{热力学第三定律}\label{2.5.4}
和内能、焓变等状态函数不同,物质的熵变是有绝对值的,\textbf{热力学第三定律}规定了熵值的基准。
\begin{formula}
    \textbf{热力学第三定律}:
    \[S(0\ \mathrm{K},\text{完美晶体})=0\]
\end{formula}\par
有了这个基准,我们就可以用一系列过程计算出物质的\textbf{标准摩尔熵}$S_\mathrm{m}^0(T)$,方法如下:
\begin{enumerate}
    \item 确定从$0\ \mathrm{K}$下的完美晶体到物质需要经历多少单纯$pVT$变化过程和相变;
    \item 若经历单纯$pVT$变化过程,用\autoref{2.5.1}中的计算方法计算出熵变,主要用恒压可逆过程;
    \item 若经历相变过程,用\autoref{2.5.3}中的计算方法计算出熵变;
    \item 最后把所有熵变累加即可;
\end{enumerate}\par
计算过程可以由\autoref{figure2.4}表示。
\begin{figure}[h]
    \centering
    \includegraphics[width=0.4\textwidth]{2.5.4规定熵.png}
    \caption{标准摩尔熵的计算}
    \label{figure2.4}
\end{figure}\par
非标准态的熵可以继续使用\autoref{2.5.1}中的方法计算,一般我们可以查询到物质在$298.15\ \mathrm{K}$下的标准摩尔熵$S_\mathrm{m}^0(298.15\ \mathrm{K})$。一般气体的标准摩尔熵远大于液体和固体。
\subsection{化学反应过程的熵变}\label{2.5.5}
和焓变一样,我们关心\textbf{标准摩尔反应熵}$\Delta_\mathrm{r}S_\mathrm{m}^0$,一般是$298.15\ \mathrm{K}$下的$\Delta_\mathrm{r}S_\mathrm{m}^0(298.15\ \mathrm{K})$。\par
和标准摩尔反应焓一样,标准摩尔反应熵不考虑物质的混合,但根据\autoref{exercise2.3},对于理想气体混合再压缩至各气体分压等于反应前分压的过程,熵变为零。如果我们考虑的“混合”前后理想气体压强或理想溶液浓度不变,那这个“混合”过程的熵变其实是零,因为混合物中各组分与混合之前没有什么区别。所以一般也可以认为标准摩尔反应熵是混合物发生反应之后的熵变。\par
焓变中的\autoref{formula2.16}盖斯定律对于熵变也有类似的计算,只要是状态函数中的广度量都有这样的计算。\par
与焓不同的是,我们可以知道熵的具体数值,因此我们可以直接用反应中物质的标准摩尔熵$S_\mathrm{m}^0$计算反应的熵变,公式和证明过程与\autoref{formula2.17}类似。
\begin{formula}
    标准摩尔熵计算标准摩尔反应熵:
    \[\Delta_\mathrm{r}S_\mathrm{m}^0=\sum\nu_iS_\mathrm{m}^0[i]\]
    \qquad 其中$i$代表反应中的物质;$\nu_i$是反应中各物质的计量数,反应物为负数,产物为正数;$S_\mathrm{m}^0[i]$是反应中各物质的标准摩尔熵。
\end{formula}\par
对于温度不是$298.15\ \mathrm{K}$的反应,反应熵也可以用类似\autoref{formula2.19}基尔霍夫公式的方法计算,由于熵为热温商,公式的形式和基尔霍夫公式略有区别。
\begin{formula}
    \textbf{任意温度$T$下反应熵}:
    \[\Delta_\mathrm{r}S_\mathrm{m}^0(T)=\Delta_\mathrm{r}S_\mathrm{m}^0(298.15\ \mathrm{K})+\int_{298.15\ \mathrm{K}}^{T}  \,\dfrac{\Delta_\mathrm{r}C_\mathrm{p,m}}{T}\diff T \]
    \qquad 其中$\Delta_\mathrm{r}C_\mathrm{p,m}=\displaystyle\sum\nu_iC_\mathrm{p,m}[i]$,即产物的恒压摩尔热容总和减去反应物的恒压摩尔热容总和。若认为热容不随温度变化,则上式可改写为:
    \[\Delta_\mathrm{r}S_\mathrm{m}^0(T)=\Delta_\mathrm{r}S_\mathrm{m}^0(298.15\ \mathrm{K})+\Delta_\mathrm{r}C_\mathrm{p,m}\ln\dfrac{T}{298.15\ \mathrm{K}}\]
\end{formula}\par
具体推导过程也和基尔霍夫公式类似。
\begin{derivation}
    \[\ce{$a$A($\alpha$) + $b$B($\beta$) -> $y$Y($\gamma$) + $z$Z($\delta$)}\]
    \begin{center}
        \includegraphics[width=0.8\textwidth]{2.5.4熵变与温度的关系.png}
    \end{center}
    \[\Delta_\mathrm{r}S_\mathrm{m}^0(T)=\Delta_\mathrm{r}S_\mathrm{m}^0(298.15\ \mathrm{K})+\Delta S_1+\Delta S_2\]
    其中$\Delta S_1$和$\Delta S_2$都可以通过摩尔热容计算。
    根据\autoref{formula2.24}恒压可逆过程熵变的计算公式:
    \begin{align*}
        \diff S_1&=\dfrac{aC_\mathrm{p,m}[\ce{A($\alpha$)}]+bC_\mathrm{p,m}[\ce{B($\beta$)}]}{T}\diff T\\
        \diff S_2&=\dfrac{yC_\mathrm{p,m}[\ce{Y($\gamma$)}]+zC_\mathrm{p,m}[\ce{Z($\delta$)}]}{T}\diff T
    \end{align*}
    两式分别积分得(注意积分上下限不同):
    \begin{align*}
    \Delta S_1&=\int_{T}^{298.15\ \mathrm{K}}  \,\dfrac{aC_\mathrm{p,m}[\ce{A($\alpha$)}]+bC_\mathrm{p,m}[\ce{B($\beta$)}]}{T}\diff T\\
    \Delta S_2&=\int_{298.15\ \mathrm{K}}^{T}  \,\dfrac{yC_\mathrm{p,m}[\ce{Y($\gamma$)}]+zC_\mathrm{p,m}[\ce{Z($\delta$)}]}{T}\diff T
    \end{align*}
    改变$\Delta S_1$积分式中的上下限得:
    \begin{align*}
    \Delta S_1&=\int_{298.15\ \mathrm{K}}^{T}  \,\dfrac{-aC_\mathrm{p,m}[\ce{A($\alpha$)}]-bC_\mathrm{p,m}[\ce{B($\beta$)}]}{T}\diff T\\
    \Delta S_2&=\int_{298.15\ \mathrm{K}}^{T}  \,\dfrac{yC_\mathrm{p,m}[\ce{Y($\gamma$)}]+zC_\mathrm{p,m}[\ce{Z($\delta$)}]}{T}\diff T
    \end{align*}
    两式相加得:
    \[\Delta S_1+\Delta S_2=\int_{298.15\ \mathrm{K}}^{T}  \,\dfrac{\sum\nu_iC_\mathrm{p,m}[i]}{T}\diff T=\int_{298.15\ \mathrm{K}}^{T}  \,\dfrac{\Delta_\mathrm{r}C_\mathrm{p,m}}{T}\diff T\]
    把结果代回$\Delta_\mathrm{r}S_\mathrm{m}^0(T)$的计算式中就可以得到\autoref{formula2.29}:
    \[\Delta_\mathrm{r}S_\mathrm{m}^0(T)=\Delta_\mathrm{r}S_\mathrm{m}^0(298.15\ \mathrm{K})+\int_{298.15\ \mathrm{K}}^{T}  \,\dfrac{\Delta_\mathrm{r}C_\mathrm{p,m}}{T}\diff T \]
\end{derivation}\par
整体上看熵变和焓变的计算大同小异,都是从定义出发一步步推导,有区别地方主要有两个:焓是恒压热,熵是热温熵,积分时有区别;焓没有具体数值,只能用生成焓代替,熵有具体数值,可以直接计算。

\end{document}