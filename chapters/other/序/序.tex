\documentclass[../../../GCET-main.tex]{subfiles}

\begin{document}

\phantomsection
\addcontentsline{toc}{chapter}{序}
\chapter*{序}

\section*{声明}
\noindent
General-Chemistry-Ever-Tracing\\
普通化学:追寻不息\\
\\
本项目中部分代码基于Linear-Algebra-Left-Undone进行开发,原作品信息如下:\\
原作品仓库:\url{https://github.com/yhwu-is/Linear-Algebra-Left-Undone}\\
原作品许可证:CC BY-NC-SA 4.0\\
\\
本项目对上述代码的使用严格遵循 CC BY-NC-SA 4.0 许可证的所有条款:
\begin{enumerate}
    \item 仅用于非商业用途;
    \item 已完整标注原作者及来源;
    \item 本项目的衍生发布版本同样采用 CC BY-NC-SA 4.0 许可证授权。
\end{enumerate}
本项目的许可证为\href{https://creativecommons.org/licenses/by-nc-sa/4.0/deed.zh}{知识共享署名-非商业性使用-相同方式共享 4.0 国际许可协议}。\\
\\
本书只在线上(CC98、QQ群、GitHub)免费发布,不会收费,不会售卖实体书,如有人在线上或线下倒卖本书,均为盗版,请读者通过下面的联系方式联系作者。

\section*{参考文献}
\begin{enumerate}
    \item 徐端钧,聂晶晶,刘清。新编普通化学 [M]. 2 版。北京:科学出版社,2012.
    \item 天津大学物理化学教研室。物理化学。上册 [M]. 7 版。北京:高等教育出版社,2024.
    \item 华彤文,王颖霞,卞江,等。普通化学原理 [M]. 4 版。北京:北京大学出版社,2013.
    \item 宋天佑,等。无机化学。上册 [M]. 4 版。北京:高等教育出版社,2019.
    \item 魏祖期,刘德育。基础化学 [M]. 8 版。北京:人民卫生出版社,2013.
    \item 严平,曹小华,黄华南,等. “稀溶液的依数性” 多维互动教学模式的探索 [J]. 大学化学,2019, 34 (7): 23-30.
\end{enumerate}

\section*{致谢}
\noindent\textbf{编写}:
\begin{itemize}
    \item 金盏花488\qquad 化学(求是科学班)2501
\end{itemize}
\textbf{审校、建议}:
\begin{itemize}
    \item Fresh Fish\qquad 化学(强基计划)2401
    \item 苏为峰\qquad 混合班2507
    \item 青芒
\end{itemize}

\section*{使用说明}
\noindent 本书分为以下几个板块:
\begin{enumerate}
    \item \autoref{0}数学物理基础:本书会用到的数学物理知识,可以提前学习,也可以在看不懂推导过程时回来参考;
    \item 正文八章:本书的核心内容,对应浙江大学普通化学教材《新编普通化学》的八章内容,部分标题后依据普通化学(H)大纲要求附带*,*越多说明知识越重要,不带*说明是补充内容;
    \item 习题与解析:正文后的习题,覆盖大部分考试题型,助力备考;
    \item 追寻章节:目前还没有,预计覆盖李浩然老师普通化学(甲)的高难度内容,其他课程可忽略或按兴趣阅读。
\end{enumerate}
本书内容比较多,可以按照读者个人的需要进行阅读:
\begin{enumerate}
    \item 零基础入门/预习普通化学:通读全书,掌握公式、知识和做题技巧,适当了解推导过程,完成练习,适当时转入第二种策略;
    \item 课内同步学习/有充足时间复习:通读全书,优先掌握公式、知识和做题技巧,熟悉推导过程,完成练习;
    \item 考前无充足时间复习:跳读,掌握公式、知识和做题技巧,忽略推导过程,按需完成练习;
    \item 完全掌握课内知识:直接看推导过程或疯狂做题。
\end{enumerate}

\section*{前言}

\subsection*{为什么要编写《普通化学:追寻不息》}

一开始我并没有编写普通化学(H)讲义的计划。我只是化学(求是科学班)2501的新生,而且我学习的是李浩然老师为求化强化特制的普通化学(甲),和普通化学(H)关系不大。我的学长易木Ewood在2025-2026学年秋冬学期担任竺可桢学院普通化学(H)课程的辅学答疑工作,学期刚开始不久,易木Ewood学长就向我们透露了编写普通化学(H)讲义的计划。但是一个学期过去好像并没有什么进度。\par
为了整理普通化学(甲)的小测卷,我学习了\LaTeX;为了预习下学期的线性代数,我阅读了《线性代数:未竟之美》,即大名鼎鼎的LALU;我又了解到普通物理学(H)也有一本辅学讲义,那只剩下普通化学(H)没有了。正值寒假,我在离校前两天有了编写普通化学(H)讲义的想法,并联系了易木Ewood学长,学长的计划是在年后开始编写。\par
我先给这份讲义取了一个和LALU类似的名字——《普通化学:追寻不息》,英文名是《General Chemistry Ever Tracing》,简称GCET,寓意永远追寻化学的奥秘。我向易木Ewood学长提议由我先准备一个模板,之后再使用GitHub合作编写后续内容,学长没有拒绝。于是在离校当天白天和回家第一天,我花费了这两天几乎所有的时间调整好了模板。然而当我向易木Ewood学长展示最后的模板时,学长突然说他其实早就搭好“框架”了,这个框架指的就是模板。我尝试进行合作,可是易木Ewood学长认为合作编写会导致“两个大脑不好搞”、“风格不一致”等原因拒绝合作,提议两个人单干,写两个版本的普通化学(H)讲义。\par

\subsection*{人真的能写过人和AI吗}

单干的后果是竞争。易木Ewood学长的课表上三休四,而我的课表是求化大一下不得不品的上五休二;易木Ewood学长已经大二了,而我还只是大一;易木Ewood学长有Gemini老师帮忙,而我只能一边和豆包辩论一边写。\par
后来才知道,易木Ewood学长说年后开始写讲义的原因是他先写了化学系有机化学III的讲义。学长的效率真的很高,他只花了两个星期就写完了91页的有机化学III讲义。Gemini可以直接帮他生成模板、编写正文、整理好所有内容并做成表格,这是豆包几辈子都不可能做到的。\par
虽然存在时间压力和竞争压力,但我仍然希望我写的内容是教材上不一定有但是对普通化学的学习有帮助的,希望我写的内容能让大家感受到化学不只是背诵和记忆,希望我写的内容能弥补学习和考试之间的壁垒,希望我写的内容不仅对普通化学(H)有帮助,也能对校内校外的大学化学、普通化学(甲)、普通化学(乙)等普通化学类课程有帮助,甚至能覆盖求化强化学习的李浩然老师的普通化学(甲)。归结起来一句话,我希望把GCET做成普通化学界的LALU。\par
马上开学,我又要开始体验没有空闲的学校生活;易木Ewood学长也在有机化学III讲义的前言中留下了“待完成普通化学讲义的编写”的字眼。虽然易木Ewood学长也是一个人,但他的时间和效率都比我更有优势。幸好我不是在孤军奋战,我还有强化班的一位学长、混合班的一位同学、我的室友愿意帮我审稿并提建议。我花费了一整个2月写完了第0到2章,覆盖了普通化学最主干的化学热力学部分。

\subsection*{第一次发布及征求援助}

鉴于接下来进度的不稳定性,我在开学前将此讲义发布在CC98,供各专业的各位前辈和同学阅读。本书仍然处于起步阶段,编写与审核的人员较少,可能存在较多笔误和不正确的内容,希望得到大家的批评与建议。\par
金盏花488欢迎各专业的各位前辈和同学参与本书的编写和审核,每一位参与编写、参与审核、提出批评和建议的前辈和同学都可以在本书开头以姓名或昵称的方式留下足迹。希望这本书能在大家的支持下不断发展完善,帮助到未来的学弟学妹们!\\

\subsection*{联系方式}
\noindent 本项目发布地址:\\
(最新版)QQ群:1084226652\\
(稳定版)GitHub:\url{https://github.com/Marigold488/General-Chemistry-Ever-Tracing}\\
\\
金盏花488\\
QQ:2637158777\\
手机号/微信:18858113677\\
邮箱:3250100737@zju.edu.cn\\

\begin{flushright}
    金盏花488\\
    化学(求是科学班)2501\\
    2026年2月28日
\end{flushright}

\end{document}